\documentclass[11pt, twocolumn]{article}
\usepackage[margin=0.7in, columnsep=0.25in, bottom=0.5in]{geometry}
\usepackage{times}
\usepackage{titlesec}
\usepackage{enumitem}
\usepackage{hyperref}
\usepackage{xcolor}

% Tight compacting
\setlist{nosep, leftmargin=*} 
\linespread{0.93} 
\setlength{\parskip}{2.5pt} 

% Tight section formatting
\titleformat{\section}{\large\bfseries}{\thesection}{1em}{}
\titlespacing*{\section}{0pt}{4pt}{1pt}

\hypersetup{
    colorlinks=true,
    linkcolor=black,
    urlcolor=blue,
    citecolor=black,
}

\pagestyle{plain} 

% Compact Bibliography
\makeatletter
\renewenvironment{thebibliography}[1]
     {\section*{References}%
      \footnotesize 
      \setlength{\itemsep}{0pt} 
      \setlength{\parskip}{0pt}
      \list{\@biblabel{\@arabic\c@enumiv}}%
           {\settowidth\labelwidth{\@biblabel{#1}}%
            \leftmargin\labelwidth
            \advance\leftmargin\labelsep
            \usecounter{enumiv}%
            \let\p@enumiv\@empty
            \renewcommand\theenumiv{\@arabic\c@enumiv}}%
      \sloppy
      \clubpenalty4000
      \@clubpenalty \clubpenalty
      \widowpenalty4000%
      \sfcode`\.\@m}
     {\def\@noitemerr
       {\@latex@warning{Empty `thebibliography' environment}}%
      \endlist}
\makeatother

\begin{document}

\twocolumn[
  \begin{@twocolumnfalse}
    \noindent
    \textbf{\LARGE AI Reading Exercise: Class 5} \hfill \textbf{Daniel Hardesty Lewis} \\
    \noindent
    \textit{From Digital to Physical Infrastructure} \hfill February 17, 2026 \\
    \textbf{PLAN A6613: AI and the Future of Cities} \hfill Spring 2026
    \vspace{3pt}
    \hrule
    \vspace{6pt}
  \end{@twocolumnfalse}
]

\section*{Tool \& Process}
I used \textbf{Claude Opus 4.6} with extended thinking via the Antigravity IDE.\footnote{Verbatim prompt log archived at \url{https://github.com/dhardestylewis/plan-a6613-ai-reading-class-3/tree/main/week5}.} My first prompt returned vague IoT sensor descriptions. I refined it to demand three \textit{named deployments} with exact cities, vendors, and funding mechanisms. A third prompt asked how each system \textit{operationally} works: what data it ingests, what it decides autonomously, and where a human remains in the loop.

\section*{Key Findings}

\textbf{1. Curb Space: Automotus in Pittsburgh.}
Pittsburgh's Smart Loading Zone program is the clearest example of AI that \textit{actively manages} a city asset. The city deployed \textbf{Automotus} computer vision cameras across 75 commercial loading zones, with cameras on streetlight poles reading license plates, enforcing tiered time limits, and automating payment without meters. According to Automotus, zone turnover rose 40\% and double-parking fell 95\% after launch. The U.S. Department of Energy funds the three-year scale-up with a \$3.8M grant, supplemented by SaaS revenue from automated payments. This is a closed operational loop: AI observes, decides, and acts.

\textbf{2. Traffic Signals: Flow Labs in North Carolina.}
North Carolina's statewide traffic signal program shows what happens when AI \textit{monitors} but does not \textit{control}. NCDOT deployed \textbf{Flow Labs} AI across more than 2,500 intersections, the largest such deployment in the US. The system ingests connected vehicle GPS data to identify signal timing problems without field studies or new hardware. But as Flow Labs' own documentation clarifies, the system only \textit{recommends} changes; a human engineer makes the final call. NCDOT funds it as a SaaS contract embedded in existing operations budgets, which means it scales without capital appropriation.

\textbf{3. Power Grid: Google Tapestry \& PJM.}
Google X's Tapestry partnership with PJM Interconnection illustrates how far ambition can outpace deployment. \textbf{Tapestry} uses DeepMind AI to model the grid topology of PJM's network, which serves 67 million people across 13 states, aiming to accelerate the years-long interconnection queue for new renewables. But unlike the first two examples, Tapestry \textit{has not yet been deployed}: it is a multi-year development partnership where Google funds AI development and PJM provides grid data. The economic logic is revealing: faster interconnection serves Google's own data center energy needs as much as it serves the public grid.

\section*{Verification}
\begin{itemize}
    \item \textbf{Automotus:} Statistics checked out. The telling gap was framing: the AI called it \textit{parking} tech when it is specifically \textbf{commercial loading} management, a key distinction for curb policy.
    \item \textbf{Flow Labs:} The AI described the system as \textit{controlling} signals. Documentation clarifies it only recommends. The human-in-the-loop distinction was entirely absent.
    \item \textbf{Tapestry:} The AI listed it alongside deployed systems without flagging that it remains a multi-year development effort, not yet operational at scale.
\end{itemize}

\section*{Critical Reflection}
The most useful finding was in what the AI \textit{failed to distinguish}. It treated all three as equivalent when they represent fundamentally different levels of readiness. A planner reading uncritically would overestimate how far AI-managed infrastructure has come. The tool is useful for assembling an initial landscape of who is doing what, but a human is essential for the harder question: \textit{how real is this?}

\begin{thebibliography}{9}

\bibitem{automotus}
Automotus. (2023). \textit{Smart Loading Zones: Pittsburgh}. \url{https://automotus.co/pittsburgh}

\bibitem{citiesToday}
Marotti, A. (2022). Pittsburgh pilots smart loading zones. \textit{Cities Today}. \url{https://cities-today.com/pittsburgh-pilots-smart-loading-zones/}

\bibitem{flowlabs}
Flow Labs. (2025). \textit{NCDOT statewide AI traffic signal deployment}. \url{https://www.flowlabs.ai}

\bibitem{statescoop}
Nyczepir, D. (2024). NC deploys AI traffic signal software statewide. \textit{StateScoop}. \url{https://statescoop.com/north-carolina-ai-traffic-signals-flow-labs/}

\bibitem{tapestry}
X, the Moonshot Factory. (2025). \textit{Tapestry}. Alphabet. \url{https://x.company/projects/tapestry/}

\bibitem{pjm}
PJM. (2025). PJM, Google multi-year AI collaboration. \url{https://www.pjm.com/about-pjm/newsroom}

\end{thebibliography}

\end{document}
