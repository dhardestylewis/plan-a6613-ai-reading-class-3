\documentclass[11pt, twocolumn]{article}
\usepackage[margin=0.7in, columnsep=0.25in, bottom=0.5in]{geometry}
\usepackage{times}
\usepackage{titlesec}
\usepackage{enumitem}
\usepackage{xurl}
\usepackage{hyperref}
\usepackage{xcolor}

% Tight compacting
\setlist{nosep, leftmargin=*} 
\linespread{0.93} 
\setlength{\parskip}{2.5pt} 

% Tight section formatting
\titleformat{\section}{\large\bfseries}{\thesection}{1em}{}
\titlespacing*{\section}{0pt}{4pt}{1pt}

\hypersetup{
    colorlinks=true,
    linkcolor=black,
    urlcolor=blue,
    citecolor=black,
    breaklinks=true,
}

\pagestyle{plain} 

% Compact Bibliography
\makeatletter
\renewenvironment{thebibliography}[1]
     {\section*{References}%
      \tiny 
      \setlength{\itemsep}{0pt} 
      \setlength{\parskip}{0pt}
      \setlength{\parsep}{0pt}
      \setlength{\topsep}{0pt}
      \list{\@biblabel{\@arabic\c@enumiv}}%
           {\settowidth\labelwidth{\@biblabel{#1}}%
            \leftmargin\labelwidth
            \advance\leftmargin\labelsep
            \usecounter{enumiv}%
            \let\p@enumiv\@empty
            \renewcommand\theenumiv{\@arabic\c@enumiv}}%
      \sloppy
      \clubpenalty4000
      \@clubpenalty \clubpenalty
      \widowpenalty4000%
      \sfcode`\.\@m}
     {\def\@noitemerr
       {\@latex@warning{Empty `thebibliography' environment}}%
      \endlist}
\makeatother

\begin{document}

\twocolumn[
  \begin{@twocolumnfalse}
    \noindent
    \textbf{\LARGE AI Reading Exercise: Class 5} \hfill \textbf{Daniel Hardesty Lewis} \\
    \noindent
    \textit{From Digital to Physical Infrastructure} \hfill February 17, 2026 \\
    \textbf{PLAN A6613: AI and the Future of Cities} \hfill Spring 2026
    \vspace{3pt}
    \hrule
    \vspace{6pt}
  \end{@twocolumnfalse}
]

\section*{Tool \& Process}
I used \textbf{Claude Opus 4.6} with extended thinking via the Antigravity IDE.\footnote{Verbatim prompt log, downloaded sources, annotations, and a claim-by-claim verification audit are archived at \url{https://github.com/dhardestylewis/plan-a6613-ai-reading-class-3/tree/main/week5}.} The AI produced initial findings on \textit{AI managing physical infrastructure}. I then directed over \textbf{140 iterative corrections}, each timestamped: downloading vendor, government, and independent sources; cross-checking statistics across stakeholders; and rewriting every unsupported claim. The most productive step was comparing \textbf{vendor-reported outcomes} against \textbf{city government data}, which revealed discrepancies the AI had not flagged.

\section*{Key Findings}

\textbf{1. Curb Space: Automotus in Pittsburgh.}
Pittsburgh's Smart Loading Zone program is the closest I found to AI that \textit{actively manages} a city asset. The city deployed \textbf{Automotus} computer vision cameras across \textbf{75 commercial loading zones}, reading license plates, enforcing tiered time limits, and automating payment without meters. Funded by a \textbf{\$3.8 million DOE grant} (Lazo, 2021) and SaaS revenue, what makes this case distinct is the \textit{closed operational loop}: the AI observes, decides, and acts without a human in between. Yet the numbers do not agree. Automotus claims \textbf{40\% higher turnover} and \textbf{95\% less double-parking}; the city reports \textbf{70\% turnover} and only \textbf{40\% less} double-parking (Automotus, 2023; City of Pittsburgh DOMI, 2024). Both parties have reason to frame results favorably, making independent verification essential.

\textbf{2. Traffic Signals: Flow Labs in North Carolina.}
North Carolina's statewide traffic signal program shows what happens when AI \textit{monitors} but does not \textit{control}. NCDOT deployed \textbf{Flow Labs} AI across \textbf{2,500 intersections} in July 2025, the largest such deployment in the U.S.\ (Nyczepir, 2024). The system uses connected vehicle GPS data to flag signal timing problems without field studies. Procured as a \textbf{SaaS contract} in existing operations budgets (Nyczepir, 2024; Flow Labs, 2025), it required no capital appropriation. Yet Flow Labs' own documentation clarifies that the system only \textit{recommends} changes; a human engineer makes the final call (Flow Labs, 2025). Aaron Moody, NCDOT's assistant director of communications, confirmed that the platform ``supports data-informed decisions while \textit{maintaining oversight by engineering staff}'' (Raths, 2025).

\textbf{3. Power Grid: Google Tapestry \& PJM.}
Google X's Tapestry partnership with PJM Interconnection is the most ambitious of the three, but also \textit{the least real}. \textbf{Tapestry} uses DeepMind AI to model grid topology across PJM's \textbf{67-million-person}, 13-state network (PJM, 2025), aiming to accelerate the years-long interconnection queue for renewables. Tapestry \textit{has not yet been deployed}: it is a \textbf{multi-year development partnership} where Google funds AI and PJM provides grid data (X, 2025). Faster interconnection directly serves Google's own data centers, and as Berreby (2024) documents in \textit{Yale Environment 360}, AI data centers are themselves a \textbf{major driver of the demand} straining the grid; Google is partly solving a problem its own infrastructure creates.

\section*{Verification and Reflection}
In all three cases the AI made the same error: it took each system at face value without interrogating stakeholder interests, autonomy, or deployment status. For Automotus, it missed the vendor-versus-city statistical discrepancy and conflated \textbf{parking} with \textbf{commercial loading} management. For Flow Labs, it described the system as \textit{controlling} signals when three sources confirm it only \textit{recommends} changes. For Tapestry, it equated a pre-deployment partnership with operational systems and noted Google's energy needs only in passing, without framing the conflict of interest or citing independent reporting.

The AI was a useful starting point for assembling an inventory of \textit{who is doing what}, but it required extensive correction to reach a product a planner could trust. Left unchecked, a reader would have concluded that all three systems \textit{actively control} city assets, when in fact only one does, one merely advises, and one does not yet exist. The harder questions, \textit{how real is this}, \textit{whose numbers are these}, and \textit{who benefits}, were invisible to the AI and required cross-checking every statistic against multiple stakeholders' accounts.

\begin{thebibliography}{9}

\bibitem{automotus}
Automotus. (2023). \textit{Smart loading zones: Pittsburgh}. Retrieved February 17, 2026, from \url{https://automotus.co/pittsburgh}

\bibitem{yaleE360}
Berreby, D. (2024, February 6). As use of A.I. soars, so does the energy and water it requires. \textit{Yale Environment 360}. \url{https://e360.yale.edu/features/artificial-intelligence-climate-energy-emissions}

\bibitem{pittsburghDOMI}
City of Pittsburgh, Department of Mobility and Infrastructure. (2024). \textit{Smart loading zones}. City of Pittsburgh. \url{https://pittsburghpa.gov/domi/smart-loading-zones}

\bibitem{flowlabs}
Flow Labs. (2025). \textit{NCDOT statewide AI traffic signal deployment}. \url{https://www.flowlabs.ai}

\bibitem{doeVTO}
Lazo, A. (2021, August 12). Pittsburgh gets \$3.8 million from feds for smart loading zones. \textit{Pittsburgh Post-Gazette}. \url{https://www.post-gazette.com/business/development/2021/08/12/}

\bibitem{statescoop}
Nyczepir, D. (2024, July 16). NC deploys AI traffic signal software statewide. \textit{StateScoop}. \url{https://statescoop.com/north-carolina-ai-traffic-signals-flow-labs/}

\bibitem{pjm}
PJM Interconnection. (2025). \textit{PJM, Google announce multi-year AI collaboration} [Press release]. \url{https://www.pjm.com/about-pjm/newsroom}

\bibitem{govtech}
Raths, D. (2025, January 14). North Carolina taps AI to help manage traffic signals statewide. \textit{Government Technology}. \url{https://www.govtech.com/transportation/north-carolina-taps-ai-to-help-manage-traffic-signals-statewide}

\bibitem{tapestry}
X, the Moonshot Factory. (2025). \textit{Tapestry}. Alphabet. Retrieved February 17, 2026, from \url{https://x.company/projects/tapestry/}

\end{thebibliography}

\end{document}
