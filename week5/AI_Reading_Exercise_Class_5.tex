\documentclass[11pt, twocolumn]{article}
\usepackage[margin=0.7in, columnsep=0.25in, bottom=0.5in]{geometry}
\usepackage{times}
\usepackage{titlesec}
\usepackage{enumitem}
\usepackage{xurl}
\usepackage{hyperref}
\usepackage{xcolor}

% Tight compacting
\setlist{nosep, leftmargin=*} 
\linespread{0.93} 
\setlength{\parskip}{2.5pt} 

% Tight section formatting
\titleformat{\section}{\large\bfseries}{\thesection}{1em}{}
\titlespacing*{\section}{0pt}{4pt}{1pt}

\hypersetup{
    colorlinks=true,
    linkcolor=black,
    urlcolor=blue,
    citecolor=black,
    breaklinks=true,
}

\pagestyle{plain} 

% Compact Bibliography
\makeatletter
\renewenvironment{thebibliography}[1]
     {\section*{References}%
      \scriptsize 
      \setlength{\itemsep}{0pt} 
      \setlength{\parskip}{0pt}
      \list{\@biblabel{\@arabic\c@enumiv}}%
           {\settowidth\labelwidth{\@biblabel{#1}}%
            \leftmargin\labelwidth
            \advance\leftmargin\labelsep
            \usecounter{enumiv}%
            \let\p@enumiv\@empty
            \renewcommand\theenumiv{\@arabic\c@enumiv}}%
      \sloppy
      \clubpenalty4000
      \@clubpenalty \clubpenalty
      \widowpenalty4000%
      \sfcode`\.\@m}
     {\def\@noitemerr
       {\@latex@warning{Empty `thebibliography' environment}}%
      \endlist}
\makeatother

\begin{document}

\twocolumn[
  \begin{@twocolumnfalse}
    \noindent
    \textbf{\LARGE AI Reading Exercise: Class 5} \hfill \textbf{Daniel Hardesty Lewis} \\
    \noindent
    \textit{From Digital to Physical Infrastructure} \hfill February 17, 2026 \\
    \textbf{PLAN A6613: AI and the Future of Cities} \hfill Spring 2026
    \vspace{3pt}
    \hrule
    \vspace{6pt}
  \end{@twocolumnfalse}
]

\section*{Tool \& Process}
I used \textbf{Claude Opus 4.6} with extended thinking via the Antigravity IDE.\footnote{Verbatim prompt log, downloaded sources, annotations, and a claim-by-claim verification audit are archived at \url{https://github.com/dhardestylewis/plan-a6613-ai-reading-class-3/tree/main/week5}.} The AI generated an initial set of findings on the topic \textit{``AI managing physical infrastructure.''} I then directed over \textbf{30 iterative corrections}, each logged with a timestamp: downloading vendor, government, and independent sources; cross-checking statistics across stakeholders; and rewriting every claim that lacked a citation or relied on a single account. The most productive step was comparing \textbf{vendor-reported outcomes} against \textbf{city government data}, which revealed discrepancies the AI had not flagged and that neither source had an incentive to disclose.

\section*{Key Findings}

\textbf{1. Curb Space: Automotus in Pittsburgh.}
Pittsburgh's Smart Loading Zone program is the closest I found to AI that \textit{actively manages} a city asset. The city deployed \textbf{Automotus} computer vision cameras across \textbf{75 commercial loading zones}, with cameras on streetlight poles reading license plates, enforcing tiered time limits, and automating payment without meters (Automotus, 2023). Automotus claims \textbf{40\% higher zone turnover} and \textbf{95\% less double-parking} (Automotus, 2023), but the City of Pittsburgh's own pilot data reports a \textbf{70\% turnover increase} and a \textbf{60\% drop} in average park duration (City of Pittsburgh DOMI, 2024). The discrepancy matters because both parties have reason to frame results favorably. The U.S. Department of Energy funded the three-year scale-up with a \textbf{\$3.8 million grant} through its Vehicle Technologies Office in August 2021 (DOE VTO, 2021), supplemented by SaaS revenue from automated payments. What makes this case distinct is the \textit{closed operational loop}: the AI observes, decides, and acts without a human in between.

\textbf{2. Traffic Signals: Flow Labs in North Carolina.}
North Carolina's statewide traffic signal program shows what happens when AI \textit{monitors} but does not \textit{control}. NCDOT deployed \textbf{Flow Labs} AI across more than \textbf{2,500 intersections} in July 2025, the largest such deployment in the United States (Nyczepir, 2024). The system ingests connected vehicle GPS data to identify signal timing problems \textit{without field studies or new hardware} (Flow Labs, 2025). Flow Labs' own documentation clarifies that the system only \textit{recommends} changes; a human engineer makes the final call (Flow Labs, 2025). Aaron Moody, NCDOT's assistant director of communications, confirmed that the platform ``supports data-informed decisions while \textit{maintaining oversight by engineering staff}'' (Raths, 2025). NCDOT funds it as a \textbf{SaaS contract} embedded in existing operations budgets, which means it scales without capital appropriation.

\textbf{3. Power Grid: Google Tapestry \& PJM.}
Google X's Tapestry partnership with PJM Interconnection is the most ambitious of the three, but also \textit{the least real}. \textbf{Tapestry} uses DeepMind AI to model the grid topology of PJM's network, which serves \textbf{67 million people} across 13 states and the District of Columbia (PJM, 2025), aiming to accelerate the years-long interconnection queue for new renewables. Unlike the first two examples, Tapestry \textit{has not yet been deployed}: it is a \textbf{multi-year development partnership} where Google funds AI development and PJM provides grid data (X, 2025). Faster interconnection also directly serves Google's own data center energy needs. As Berreby (2024) documents in \textit{Yale Environment 360}, AI data centers are themselves a \textbf{major driver of the electricity demand} straining the grid, which means Google is partly solving a problem its own infrastructure creates.

\section*{Verification}
\begin{itemize}
    \item \textbf{Automotus:} Vendor and city statistics did not match. Automotus reports 40\% turnover and 95\% less double-parking (Automotus, 2023); the City of Pittsburgh reports 70\% turnover and \textit{only} 40\% less double-parking (City of Pittsburgh DOMI, 2024). The AI did not flag this discrepancy, nor did it distinguish \textit{parking} from \textbf{commercial loading} management, a difference that matters for how planners allocate curb space.
    \item \textbf{Flow Labs:} Three independent sources contradict the AI's claim that Flow Labs \textit{controls} signals. Flow Labs' documentation (2025) says the system \textit{recommends} changes. StateScoop (Nyczepir, 2024) describes it as a monitoring platform. NCDOT spokesperson Aaron Moody confirmed that ``engineering staff retain oversight'' (Raths, 2025). The human-in-the-loop distinction was entirely absent from the AI's output.
    \item \textbf{Tapestry:} The AI presented Tapestry as comparable to the first two deployed systems, but the X project page (2025) describes it as a multi-year development partnership that is \textit{not yet operational}. Neither the AI nor Google's own materials acknowledged that faster interconnection directly benefits Google's data centers, a conflict of interest documented by Berreby (2024) in \textit{Yale Environment 360}.
\end{itemize}

\section*{Critical Reflection}
The most useful finding was in what the AI \textit{failed to distinguish}. It treated all three systems as equivalent when they represent fundamentally different levels of \textbf{autonomy} and \textbf{readiness}. A planner reading the AI's output uncritically would overestimate how far AI-managed infrastructure has come. The tool is useful for assembling an initial inventory of \textit{who is doing what}, but a human is needed for the harder question of \textit{how real is this}, and must cross-check every statistic against multiple stakeholders' accounts.

\begin{thebibliography}{9}

\bibitem{automotus}
Automotus. (2023). \textit{Smart loading zones: Pittsburgh}. Retrieved February 17, 2026, from \url{https://automotus.co/pittsburgh}

\bibitem{yaleE360}
Berreby, D. (2024, February 6). As use of A.I. soars, so does the energy and water it requires. \textit{Yale Environment 360}. \url{https://e360.yale.edu/features/artificial-intelligence-climate-energy-emissions}

\bibitem{pittsburghDOMI}
City of Pittsburgh, Department of Mobility and Infrastructure. (2024). \textit{Smart loading zones}. City of Pittsburgh. \url{https://pittsburghpa.gov/domi/smart-loading-zones}

\bibitem{flowlabs}
Flow Labs. (2025). \textit{NCDOT statewide AI traffic signal deployment}. \url{https://www.flowlabs.ai}

\bibitem{statescoop}
Nyczepir, D. (2024, July 16). NC deploys AI traffic signal software statewide. \textit{StateScoop}. \url{https://statescoop.com/north-carolina-ai-traffic-signals-flow-labs/}

\bibitem{pjm}
PJM Interconnection. (2025). \textit{PJM, Google announce multi-year AI collaboration} [Press release]. \url{https://www.pjm.com/about-pjm/newsroom}

\bibitem{govtech}
Raths, D. (2025, January 14). North Carolina taps AI to help manage traffic signals statewide. \textit{Government Technology}. \url{https://www.govtech.com/transportation/north-carolina-taps-ai-to-help-manage-traffic-signals-statewide}

\bibitem{doeVTO}
Lazo, A. (2021, August 12). Pittsburgh gets \$3.8 million from feds for smart loading zones. \textit{Pittsburgh Post-Gazette}. \url{https://www.post-gazette.com/business/development/2021/08/12/}

\bibitem{tapestry}
X, the Moonshot Factory. (2025). \textit{Tapestry}. Alphabet. Retrieved February 17, 2026, from \url{https://x.company/projects/tapestry/}

\end{thebibliography}

\end{document}
