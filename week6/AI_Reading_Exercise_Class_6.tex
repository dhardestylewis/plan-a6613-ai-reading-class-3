\documentclass[11pt, twocolumn]{article}
\usepackage[margin=0.65in, columnsep=0.25in, bottom=0.45in]{geometry}
\usepackage{newtxtext}
\usepackage{titlesec}
\usepackage{enumitem}
\usepackage{xurl}
\usepackage{hyperref}
\usepackage{xcolor}

% Tight compacting
\setlist{nosep, leftmargin=*} 
\linespread{0.93} 
\setlength{\parskip}{2.5pt} 
\renewcommand{\footnotesize}{\tiny}

% Tight section formatting with visual distinction
\titleformat{\section}{\large\bfseries\scshape}{\thesection}{1em}{}[\vspace{1pt}\titlerule]
\titlespacing*{\section}{0pt}{6pt}{3pt}

\hypersetup{
    colorlinks=true,
    linkcolor=black,
    urlcolor=blue,
    citecolor=black,
    breaklinks=true,
}

\pagestyle{plain} 

% Compact Bibliography
\makeatletter
\renewenvironment{thebibliography}[1]
     {\section*{References}%
      \tiny 
      \setlength{\itemsep}{0pt} 
      \setlength{\parskip}{0pt}
      \setlength{\parsep}{0pt}
      \setlength{\topsep}{0pt}
      \list{\@biblabel{\@arabic\c@enumiv}}%
           {\settowidth\labelwidth{\@biblabel{#1}}%
            \leftmargin\labelwidth
            \advance\leftmargin\labelsep
            \usecounter{enumiv}%
            \let\p@enumiv\@empty
            \renewcommand\theenumiv{\@arabic\c@enumiv}}%
      \sloppy
      \clubpenalty4000
      \@clubpenalty \clubpenalty
      \widowpenalty4000%
      \sfcode`\.\@m}
     {\def\@noitemerr
       {\@latex@warning{Empty `thebibliography' environment}}%
      \endlist}
\makeatother

\begin{document}

\twocolumn[
  \begin{@twocolumnfalse}
    \noindent
    \textbf{\LARGE AI Reading Exercise: Class 6} \hfill \textbf{Daniel Hardesty Lewis} \\
    \noindent
    \textit{Urban Development in an Age of Automation} \hfill February 24, 2026 \\
    \textbf{PLAN A6613: AI and the Future of Cities} \hfill Spring 2026
    \vspace{3pt}
    \hrule
    \vspace{6pt}
  \end{@twocolumnfalse}
]

\section*{Tool \& Process}
I used \textbf{Claude Opus 4.6} with extended thinking via the Antigravity IDE.\footnote{Verbatim prompt log, downloaded sources, annotations, and a claim-by-claim verification audit are archived at \url{https://github.com/dhardestylewis/plan-a6613-ai-reading-class-3/tree/main/week6}.} The prompt asked for two \textit{agentic urban systems} and how their governance addresses transparency and community consent. Claude assembled useful candidates but treated vendor claims and proposed governance mechanisms as credible without independent verification. I then directed over 20 iterative corrections: downloading vendor, government, civil liberties, and independent sources; cross-checking statistics across stakeholders; and rewriting unsupported claims. The most productive step was comparing vendor-reported accuracy against independent audits, which revealed discrepancies Claude had not flagged.

\section*{Key Findings}

\textbf{\textit{1. ShotSpotter/SoundThinking in Chicago.}}
Chicago deployed SoundThinking's (formerly ShotSpotter) acoustic gunshot detection system across 117 square miles of predominantly Black and Latino neighborhoods, placing sensors on buildings and poles that use machine-learning algorithms to detect, classify, and geolocate gunfire in real-time (MacArthur Justice Center, 2022). When the system registers a potential gunshot, it autonomously alerts police dispatch within 60 seconds, triggering armed responses without any resident call or complaint. Yet the system's effectiveness does not match the vendor's claims. Chicago's Office of Inspector General analyzed over 50,000 alerts from January 2020 through May 2021 and found that only 9.1\% of police responses produced evidence of a gun-related crime (OIG Chicago, 2021). SoundThinking cites 97\% accuracy, but that metric measures whether sensors correctly identify a sound as a gunshot, not whether police find evidence of a crime upon arrival. The city renewed the initial \$33 million contract multiple times without public hearings, ultimately spending over \$50 million, and sensors were placed in specific neighborhoods without residents' input; the proprietary algorithm's inner workings remained undisclosed (ACLU, 2021). Mayor Brandon Johnson ended the contract in September 2024 after sustained public pressure (South Side Weekly, 2024), but the pattern of deployment without consent had persisted for nearly a decade.

\textbf{\textit{2. Sidewalk Labs Quayside in Toronto.}}
Google's Sidewalk Labs proposed Quayside, a 12-acre neighborhood on Toronto's eastern waterfront, as a testbed for AI-driven urban management: adaptive traffic signals, automated waste collection, and real-time energy optimization, all fed by continuous data collection from cameras, environmental sensors, and mobile devices (Sidewalk Labs, 2019). The governance model centered on a proposed ``Civic Data Trust'' to set rules for data use, but three structural failures undermined it. The project expanded from 12 to 190 acres without public mandate, raising questions about scope creep and land control (Sauter, 2019). Ann Cavoukian, Ontario's former Privacy Commissioner and the architect of ``Privacy by Design,'' resigned as advisor in October 2018 when Sidewalk Labs could not guarantee that all third-party data would be de-identified at collection, making privacy protections voluntary rather than enforceable (Cavoukian, 2018). Entrepreneur Saadia Muzaffar separately resigned from Waterfront Toronto's advisory panel, citing ``apathy and a lack of leadership regarding shaky public trust'' over data and privacy (Muzaffar, 2018). Sidewalk Labs withdrew entirely in May 2020, citing the pandemic, but the project had become what open-government advocate Bianca Wylie characterized as a corporate experiment in urban data control conducted without meaningful community consent (Wylie, 2020).

\section*{Verification and Reflection}
Both cases exposed the same blind spot: Claude took each system at face value without interrogating who controlled the data, who was excluded from decision-making, or whether governance structures had any enforcement power. For ShotSpotter, it described the system as an effective public safety tool without surfacing the 9.1\% evidence rate or the absence of community consent. For Quayside, it presented the Civic Data Trust as a credible governance mechanism without noting that the project's own privacy advisors resigned over its inadequacy. The questions that mattered most to a planner---whose data is this, who consented, and who benefits---were invisible to the AI and required cross-checking vendor claims against government audits, civil liberties organizations, and independent journalism.

\begin{thebibliography}{9}
\bibitem{aclu} ACLU. (2021, August 24). Four problems with the ShotSpotter gunshot detection system. \textit{American Civil Liberties Union}. \url{https://www.aclu.org/news/privacy-technology/four-problems-with-the-shotspotter-gunshot-detection-system}
\bibitem{cavoukian} Cavoukian, A. (2018, October 23). Statement on resignation from Sidewalk Labs advisory role. \textit{Global News}. \url{https://globalnews.ca/news/4583368/ann-cavoukian-sidewalk-labs-resignation/}
\bibitem{macarthur} MacArthur Justice Center. (2022, July 21). \textit{Williams v. City of Chicago}: Federal lawsuit challenges ShotSpotter technology. \textit{AP News}. \url{https://apnews.com/article/shotspotter-chicago-police-technology-lawsuit-90d3f8bf1e6e41548e4e0e2b8bf0c6a2}
\bibitem{muzaffar} Muzaffar, S. (2018, October 5). Resignation from Waterfront Toronto Digital Strategy Advisory Panel. \textit{CTV News}. \url{https://www.ctvnews.ca/sci-tech/another-adviser-quits-waterfront-toronto-board-over-sidewalk-labs-partnership-1.4122348}
\bibitem{oig} Office of Inspector General, City of Chicago. (2021, August 24). \textit{The Chicago Police Department's use of ShotSpotter technology}. \url{https://igchicago.org/publications/the-chicago-police-departments-use-of-shotspotter-technology/}
\bibitem{sauter} Sauter, M. (2019, June 24). Sidewalk Labs' plan for Toronto suggests ``Google''-scale ambitions. \textit{The Globe and Mail}. \url{https://www.theglobeandmail.com/business/article-sidewalk-labs-plan-for-toronto/}
\bibitem{sidewalk} Sidewalk Labs. (2019). \textit{Toronto Tomorrow: Master Innovation and Development Plan}. Volumes 1--3. \url{https://www.sidewalklabs.com/toronto}
\bibitem{ssw} South Side Weekly. (2024, September 24). ShotSpotter contract ends in Chicago. \textit{South Side Weekly}. \url{https://southsideweekly.com/}
\bibitem{wylie} Wylie, B. (2020, May 7). Sidewalk Toronto is dead. Long live Sidewalk Toronto. \textit{biancawylie.com}. \url{https://biancawylie.com/writings/sidewalk-toronto-is-dead}
\end{thebibliography}

\end{document}
