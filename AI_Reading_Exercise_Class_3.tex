\documentclass[11pt, twocolumn]{article}
\usepackage[margin=0.75in, columnsep=0.25in]{geometry}
\usepackage{times}
\usepackage{titlesec}
\usepackage{enumitem}
\usepackage{hyperref}
\usepackage{xcolor}
\usepackage{fancyhdr}

% Compact list spacing
\setlist{itemsep=1pt, topsep=1pt, leftmargin=*}
\linespread{0.98}
\setlength{\parskip}{2pt}

% Section formatting
\titleformat{\section}{\large\bfseries}{\thesection}{1em}{}
\titlespacing*{\section}{0pt}{8pt}{4pt}

\hypersetup{
    colorlinks=true,
    linkcolor=black,
    urlcolor=blue,
    citecolor=black,
}

% Remove header from first page to avoid repetition
\pagestyle{plain} 

% Fix referencing: Define fully detailed bibliography
\makeatletter
\renewenvironment{thebibliography}[1]
     {\section*{References}%
      \footnotesize 
      \list{\@biblabel{\@arabic\c@enumiv}}%
           {\settowidth\labelwidth{\@biblabel{#1}}%
            \leftmargin\labelwidth
            \advance\leftmargin\labelsep
            \usecounter{enumiv}%
            \let\p@enumiv\@empty
            \renewcommand\theenumiv{\@arabic\c@enumiv}}%
      \sloppy
      \clubpenalty4000
      \@clubpenalty \clubpenalty
      \widowpenalty4000%
      \sfcode`\.\@m}
     {\def\@noitemerr
       {\@latex@warning{Empty `thebibliography' environment}}%
      \endlist}
\makeatother

\begin{document}

\twocolumn[
  \begin{@twocolumnfalse}
    \noindent
    \textbf{\LARGE AI Reading Exercise: Class 3} \hfill \textbf{Daniel Hardesty Lewis} \\
    \noindent
    \textit{The Economic Impact of Urban Technologies} \hfill February 3, 2026 \\
    \textbf{PLAN A6613: AI and the Future of Cities} \hfill Spring 2026
    \vspace{5pt}
    \hrule
    \vspace{15pt}
  \end{@twocolumnfalse}
]

\section*{Tool \& Process}
\textbf{AI Tool Used:} Google Gemini.

\noindent
\textbf{Process:} I utilized iterative prompting to move beyond generic answers. The most effective strategy was requesting "distinct measurement methodologies" rather than general impacts. The core prompt was: \textit{"What are three specific observational methodologies cities use to measure economic impact, excluding simple GDP metrics?"} This successfully flagged the ISO standard and Big Data approaches quickly.

\section*{Key Findings: Distinct Methodologies}
\begin{itemize}
    \item \textbf{Standardized Indicator Frameworks:} Cities rely on standards like \textbf{ISO 37122} to benchmark performance. Unlike general economic stats, this includes specific urban-tech metrics like "percentage of labor force in ICT" and "number of startups per capita," allowing comparison between global smart cities \cite{iso37122}.

    \item \textbf{Life Cycle Cost-Benefit Analysis (LCCBA):} For physical infrastructure, cities use LCCBA to justify high initial costs. This method anticipates long-term savings in maintenance and indirect "social costs" (e.g., traffic disruption avoided). New York City's analysis of "utilidors" (utility tunnels) is a key example, quantifying the value of accessible subsurface infrastructure \cite{nycutilidors}.

    \item \textbf{Big Data \& Digital Footprints:} Moving beyond annual censuses, cities now use "nowcasting" with digital data. Methodologies involve aggregating anonymized private-sector data (cell phone pings, credit card transactions, LinkedIn hiring shifts) to measure economic activities in real-time \cite{worldbankdata}.
\end{itemize}

\section*{Verification ("Human-in-the-Loop")}
\textbf{Methodology:} Verified all sources by locating the original PDF reports.
\begin{itemize}
    \item \textbf{ISO 37122:} Confirmed standard exists and covers "smart city" indicators specifically (distinct from general sustainability) \cite{iso37122}.
    \item \textbf{NYC Utilidors:} Located the specific DDC working group report that details the LCCBA framework \cite{nycutilidors}.
    \item \textbf{World Bank:} Verified the 2017 Innovation Challenge report discusses these specific digital proxy methods \cite{worldbankdata}.
\end{itemize}

\noindent
\textbf{Hallucination/Accuracy Check:} The AI was generally accurate but lazy with citations. It suggested a non-specific "Global Smart City Report 2024" which does not exist as a primary source. I had to manually guide it to the specific \textbf{ISO 37122} standard and find the direct \textbf{Town+Gown} PDF.

\section*{Critical Reflection}
The AI excels as a "methodology map" generator—it quickly identified the three correct analytical "buckets" (Standards, Financial Analysis, Big Data). However, it failed to act as a reliable librarian. Its citations were either generic or landing pages rather than direct documents. For a planner, it is a useful brainstorming tool but requires verification of every specific claim and source.

% References in APA-style format (author, year, title, publisher, link)
\begin{thebibliography}{9}

\bibitem{iso37122}
International Organization for Standardization (ISO). (2019). \textit{ISO 37122:2019 Sustainable cities and communities — Indicators for smart cities}. Geneva, Switzerland: ISO. Available at: \url{https://www.iso.org/standard/69050.html} [Accessed Feb. 3, 2026].

\bibitem{nycutilidors}
NYC Department of Design and Construction (DDC). (2018). \textit{Town+Gown:NYC Utilidor Working Group Resources}. New York, NY: City of New York. Available at: \url{https://www1.nyc.gov/site/ddc/about/town-gown-working-groups.page} [Accessed Feb. 3, 2026].

\bibitem{worldbankdata}
World Bank Group. (2017). \textit{Big Data Innovation Challenge: Measuring Economic Activity}. Washington, D.C.: World Bank. Available at: \url{https://www.worldbank.org/en/topic/bigdata} [Accessed Feb. 3, 2026].

\end{thebibliography}

\end{document}
