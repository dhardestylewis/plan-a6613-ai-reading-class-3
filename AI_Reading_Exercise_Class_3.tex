\documentclass[11pt, twocolumn]{article}
\usepackage[margin=0.75in, columnsep=0.25in]{geometry}
\usepackage{times}
\usepackage{titlesec}
\usepackage{enumitem}
\usepackage{hyperref}
\usepackage{xcolor}
\usepackage{fancyhdr}

% List spacing
\setlist{itemsep=1pt, topsep=1pt, leftmargin=*}
\linespread{0.98} % Slight squeeze

% Section formatting
\titleformat{\section}{\large\bfseries}{\thesection}{1em}{}
\titlespacing*{\section}{0pt}{6pt}{3pt}

\hypersetup{
    colorlinks=true,
    linkcolor=black,
    urlcolor=blue,
    citecolor=black,
}

% Header/Footer
\pagestyle{fancy}
\fancyhf{}
\lhead{\textbf{AI Reading Exercise: Class 3}}
\rhead{Daniel Hardesty Lewis}
\lfoot{PLANA6613\_001\_2026\_1}
\rfoot{February 3, 2026}
\renewcommand{\headrulewidth}{0.4pt}
\renewcommand{\footrulewidth}{0.4pt}

\begin{document}

\twocolumn[
  \begin{@twocolumnfalse}
    \noindent
    \textbf{\LARGE AI Reading Exercise: Class 3} \hfill \textbf{Daniel Hardesty Lewis} \\
    \noindent
    \textit{The Economic Impact of Urban Technologies} \hfill February 3, 2026
    \vspace{5pt}
    \hrule
    \vspace{10pt}
  \end{@twocolumnfalse}
]

\section*{Tool \& Process}
\textbf{AI Tool Used:} Google Gemini.

\noindent
\textbf{Process:} Use of iterative prompting was necessary to move beyond generic answers. The most effective strategy was requesting "distinct measurement methodologies" rather than general impacts, specifically asking for: \textit{"observational methodologies cities use to measure economic impact, excluding simple GDP metrics."} This successfully flagged standards and big data approaches.

\section*{Key Findings: Distinct Methodologies}
\begin{itemize}
    \item \textbf{Standardized Indicator Frameworks:} Cities rely on the \textbf{ISO 37122} standard ("Indicators for smart cities") to benchmark performance. Unlike general economic stats, this includes specific urban-tech metrics like "percentage of labor force in ICT" and "number of startups per capita," allowing comparison between global smart cities \cite{iso37122}.

    \item \textbf{Life Cycle Cost-Benefit Analysis (LCCBA):} For physical infrastructure, cities use LCCBA to justify high initial costs. This method anticipates long-term savings in maintenance and indirect "social costs" (e.g., traffic disruption avoided). NYC's analysis of "utilidors" (utility tunnels) is a key example, quantifying the value of accessible subsurface infrastructure \cite{nycutilidors}.

    \item \textbf{Big Data \& Digital Footprints:} Moving beyond annual censuses, cities now use "nowcasting" with digital data. Methodologies involve aggregating anonymized private-sector data (cell phone pings, credit card transactions, LinkedIn hiring shifts) to measure economic activities in real-time, particularly for the informal economy \cite{worldbankdata}.
\end{itemize}

\section*{Verification ("Human-in-the-Loop")}
\textbf{Real-World Citations:} Supported by ISO 37122 \cite{iso37122}, NYC's Utilidor Cost-Benefit Analysis \cite{nycutilidors}, and World Bank Big Data methodologies \cite{worldbankdata}.

\noindent
\textbf{Hallucination/Accuracy Check:} The AI was generally accurate but lazy with citations. It initially suggested a non-specific "Global Smart City Report 2024" which does not exist as a primary source. I had to manually guide it to the specific \textbf{ISO 37122} standard (distinguishing it from the older 37120) and find the direct \textbf{Town+Gown} PDF, as the AI only provided the agency's landing page.

\section*{Critical Reflection}
The AI excels as a "methodology map" generator—it quickly identified the three correct analytical "buckets" (Standards, Financial Analysis, Big Data). However, it failed to act as a reliable librarian. Its citations were either generic or landing pages rather than direct documents. For a planner, it is a useful brainstorming tool but requires verification of every specific claim and source.

% References
\bibliographystyle{IEEEtran}
% refname fix: standard article usually uses \refname, IEEEtran might use different hooks depending on package load order.
% Just resetting command to empty to let bibtex handle or manually headers if needed, but standard IEEEtran usually adds its own header.
% We will manually added the header above if needed, but standard IEEEtran usually works. 
% Actually, standard article class + IEEEtran style:
\makeatletter
\renewcommand{\@biblabel}[1]{[#1]}
\makeatother

% reducing spacing in bibliography
\let\OLDthebibliography\thebibliography
\renewcommand\thebibliography[1]{
  \OLDthebibliography{#1}
  \setlength{\parskip}{0pt}
  \setlength{\itemsep}{0pt plus 0.3ex}
}

\bibliography{references} % We need a bib file or embedded bib. Using embedded for simplicity in this context implies \begin{thebibliography} usually.

\begin{thebibliography}{9}

\bibitem{iso37122}
ISO, ``ISO 37122:2019 Sustainable cities and communities,'' 2019. [Online]. \url{https://www.iso.org/standard/69050.html}

\bibitem{nycutilidors}
NYC DDC, ``Town+Gown: Utilidor Working Group,'' n.d. [Online]. \url{https://www1.nyc.gov/site/ddc/about/town-gown-working-groups.page}

\bibitem{worldbankdata}
World Bank, ``Big Data for Sustainable Development,'' 2017. [Online]. \url{https://www.worldbank.org/en/topic/bigdata}

\end{thebibliography}

\end{document}
